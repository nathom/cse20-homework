\documentclass{article}
\usepackage{amsmath,amsfonts}
\newcommand{\R}{\mathbb{R}}
\newcommand*\xor{\oplus}
\newcommand{\Z}{\mathbb{Z}}
\newcommand{\N}{\mathbb{N}}
\newcommand{\Q}{\mathbb{Q}}

\title{Homework 4}
\author{
	Nathaniel Thomas\\
	\texttt{A17069898}
}
\begin{document}
\maketitle
\begin{enumerate}
	\item \begin{enumerate}
		      \item $$
			            \forall x \in \Q (R(x))
		            $$

		      \item $$
			            \exists x (F(x))
		            $$

		      \item $$
			            \forall x \in \Z (\lnot A(x))
		            $$

		      \item $$
			            \forall y \in \Z (M(0,y))
		            $$

		      \item $$
			            \forall x (BM(x) \land BE(x))
		            $$

		      \item $$
			            \exists x \in \R (0 < x < 2 \land T(x))
		            $$

		      \item $$
			            \forall x \in \Z (Z(x^2))
		            $$

		      \item $$
			            \forall x \in \Z (\lnot N(x^2))
		            $$

		      \item $$
			            \exists x \in \R (\lnot Q(x))
		            $$

		      \item $$
			            (\forall x \in \R)(\exists y)(Z(y) \land LTE(y,x) \land LT(x, y+1))
		            $$
	      \end{enumerate}

	\item \begin{enumerate}
		      \item \begin{enumerate}
			            \item True
			            \item False. Let $x = 1$. By the definition of $\Z^+$, there is no
			                  element that is less than 1.
			            \item True
			            \item True
		            \end{enumerate}
		      \item \begin{enumerate}
			            \item False. Let $x=2$ and $y=3$. By the definition of $\Z$, there is
			                  no element $z$ such that $x < z < y$. This means the proposition
			                  will be $(T \to F) \equiv F$, so the statement is false.
			            \item True
			            \item True
			            \item True
		            \end{enumerate}
		      \item \begin{enumerate}
			            \item True
			            \item True
			            \item True
			            \item True
		            \end{enumerate}
		      \item \begin{enumerate}
			            \item False. Let $x=2$. There is no $y$ in $\Z^+$ such that $y^2=2$
			                  because $\sqrt{2} \notin \Z$.
			            \item False. Let $x = -\sqrt 2$. This implies that $(x = y^2)$ is
			                  false because there does not exist an $n \in \R$ such that $n^2$
			                  is negative, by the definition of squaring. Since
			                  $\R - \Q \subseteq \R$, the statement is false.
			            \item False. Let $x = -1$. This implies that $(x = y^2)$ is
			                  false because there does not exist an $n \in \R$ such that $n^2$
			                  is negative, by the definition of squaring. This means the
			                  statement is false.
			            \item True.
		            \end{enumerate}
	      \end{enumerate}
	\item \begin{enumerate}
		      \item Given functions $f$ and $g$, and positive integers $n$ and $k$,
		            $f(n)$ will be less than a real-numbered constant times
		            $g(n)$ if $n$ is greater or equal to a lower bound $k$.

		      \item $$
			            \lnot O(f,g) = (\forall C \in \R^+)(\forall k \in Z^+)(
			            \exists n \in Z^+) (n \ge k \land f(n) > Cg(n))
		            $$

		      \item Let $C = 1, k = 4$. We can choose $n = 5$. This means $n \ge k$ is
		            True, and $f(n) = 64 \le 1 \cdot g(n) = 64$, so $f(n) \le Cg(n)$ is
		            satisfied. This means the statement $(n \ge k \to f(n) \le C g(n))$
		            is equivalent to $(T \to T) \equiv T$.

		      \item There exists a function $f$ in $V$ for every function $g$ in $V$
		            such that $f(n)$ is bounded by $Cg(n)$ for all $n \ge$
		            a positive integer $k$, with $n$ being a positive integer
		            and $C$ being a real number.

		            Negation:

		            $$
			            (\forall f \in V)(\exists g \in V) (\lnot O(f,g))
		            $$

		      \item For all functions $f$ and $g$ in $V$, $C g(n)$ is an upper bound
		            for $f(n)$ if and only if $Cf(n)$ is a lower bound for $g(n)$ for
		            all $n$ greater than a lower bound $k$, with $n,k$ being positive
		            integers and $C$ being a real number.

		            Negation:

		            $$
			            (\exists f \in V)(\exists g \in V)(O(f,g) \xor \Omega(g,f))
		            $$

		      \item For all functions $f,g$ and $h$ in $V$, if $C_1 g(n)$ is an upper
		            bound for $f(n)$ and $C_2 h(n)$ is an upper bound for $g(n)$, then
		            $C_3 h(n)$ is an upper bound for $f(n)$ for all $n$ greater than $k$,
		            with $n,k$ being positive integers, and $C_1,C_2, C_3$ real numbers.

		            Negation:

		            $$
			            (\exists f \in V)(\exists g \in V)(\exists h \in V)
			            ((O(f,g) \land O(g,h)) \land \lnot O(f,h))
		            $$

		      \item There exist functions $f$ and $g$ in $V$ such that $f(n)$ will be
		            less than a real-numbered constant times $g(n)$ if $n$ is
		            greater or equal to a lower bound $k$ and $g(n)$ will be
		            less than a real-numbered constant times $f(n)$ if $n$ is
		            greater or equal to a lower bound $k$, and $f$ is not equal to
		            $g$ with $n,k$ being positive integers.

		            Negation:

		            $$
			            (\forall f \in V)(\forall g \in V)
			            (\lnot O(f,g) \lor \lnot \Omega(f,g) \lor (f = g))
		            $$
	      \end{enumerate}
\end{enumerate}
\end{document}
