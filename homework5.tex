\documentclass{article}
\newcommand{\R}{\mathbb{R}}
\newcommand*\xor{\oplus}
\newcommand{\Z}{\mathbb{Z}}
\newcommand{\N}{\mathbb{N}}
\newcommand{\Q}{\mathbb{Q}}
\newcommand{\mult}{\textit{Multiples}}
\usepackage{amsmath,amsfonts}
\begin{document}
\begin{enumerate}
	\item \begin{enumerate}
		      \item We can use the counterexample $x=6$. $x$ does not divide
		            4, but it is even.
		      \item We can use the counterexample $a=1, b=2, c=3$. $a+b = 3$ and
		            $b + c = 5$, but $a+c = 4$, which is even.

		      \item Towards a proof by universal generalization. We want to show
		            $$
			            \forall r \in \R (r^2 + r \text{ is even} \to r \notin \Z)
		            $$

		            We can rewrite the statement as

		            $$
			            r(r + 1)
		            $$

		            if $r \in \Z$, the statement must be even, since an even
		            times an odd number is even. This proves the contrapositive

		            $$
			            r \in \Z \to r^2 + r \text{ is even}
		            $$

		      \item We want to disprove the claim, with a proof by contradiction. Assume the following
		            is true.

		            $$
			            \exists a \in \Z^+ \exists b \in \Z^+ ( \sqrt{a + b} = \sqrt a + \sqrt b)
		            $$

		            With algebraic manipulations we get the following

		            \begin{align*}
			            (\sqrt{a + b})^2 & = (\sqrt a + \sqrt b)^2 \\
			            a + b            & = a + 2 \sqrt{ab} + b   \\
			            0                & = 2 \sqrt{ab}           \\
		            \end{align*}

		            Because $a$ and $b$ are postitive integers, the square root of
		            their product must be positive, and 2 times that quantity must be too.
	      \end{enumerate}

	\item \begin{enumerate}
		      \item We can prove the following with a witness. Let $x = 1$. This means the
		            set of the multiples of $x = \Z^+$. Since the multiples of $y \in \Z^+$ are in $\Z^+$,
		            the statement is true.

		      \item We want to show the following

		            $$
			            \exists x \forall y (x = y \lor \textit{Multiples}(x)
			            \not\subseteq \textit{Multiples}(y))
		            $$

		            Let $x = 1$. If $y = 1 = x$, then the first part of the statement is true. Let's
		            now assume $y \ne 1$. WTS

		            $$
			            y \ne 1 \land
			            \textit{Multiples}(1)
			            \not\subseteq \textit{Multiples}(y)
		            $$

		            TODO

		      \item We want to disprove, and show the following is true

		            $$
			            \exists x \exists y \forall z (\lnot (\mult(x) \cup \mult(y) = \mult(z)))
		            $$

					Let $x=2, y=3$, then we need to show that there is no $z$ such that $\mult(z)$
					contains both 2 and 3, which is in $\mult(x) \cup \mult(y)$. By the definition
					of multiplication, $z$ must be less than or equal to its lowest multiple. This leaves

					Heloo, this is a change.
					Darren was here.
					Nathan has an enormous nose.
	      \end{enumerate}
\end{enumerate}
\end{document}
