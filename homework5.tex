\documentclass{article}
\newcommand{\R}{\mathbb{R}}
\newcommand*\xor{\oplus}
\newcommand{\Z}{\mathbb{Z}}
\newcommand{\N}{\mathbb{N}}
\newcommand{\Q}{\mathbb{Q}}
\newcommand{\mult}{\textit{Multiples}}
\usepackage{amsmath,amsfonts,amssymb}
\newcommand{\qed}{\hfill\blacksquare}
\usepackage{enumerate}

\hbadness=100000000
\hfuzz=100pt
\title{Homework 5}
\author{
	Nathaniel Thomas\\
	\texttt{A17069898}
	\and
	Brandon Szeto\\
	\texttt{A17002478}
	\and
	Darren Yu\\
	\texttt{A16760914}
}
\begin{document}

\maketitle

\begin{enumerate}
	\item \begin{enumerate}
		      \item We can use the counterexample $x=6$. $x$ does not divide
		            4, but it is even.
		      \item We can use the counterexample $a=1, b=2, c=3$. $a+b = 3$ and
		            $b + c = 5$, but $a+c = 4$, which is even.

		      \item Towards a proof by universal generalization. We want to show
		            $$
			            \forall r \in \R (r^2 + r \text{ is odd} \to r \notin \Z)
		            $$

		            We can rewrite the statement as

		            $$
			            r(r + 1)
		            $$

		            \textbf{Case 1:} $r$ is even

		            If $r$ is even, then $r$ times any number will also be even.

		            \textbf{Case 2:} $r$ is odd

		            If $r$ is odd, $r+1$ is even, and their product will be even.

		            if $r \in \Z$, the statement must be even, since an even.
		            This proves the contrapositive

		            $$
			            r \in \Z \to r^2 + r \text{ is even}
		            $$

		      \item
		            We want to disprove the claim with a proof by contradiction. Assume the following
		            is true.

		            $$
			            \exists a \exists b ( \sqrt{a + b} = \sqrt a + \sqrt b)
		            $$

		            \begin{align*}
			            (\sqrt{a + b})^2 & = (\sqrt a + \sqrt b)^2 \\
			            a + b            & = a + 2 \sqrt{ab} + b   \\
			            0                & = 2 \sqrt{ab}           \\
			            0                & = ab                    \\
		            \end{align*}

		            The product of two integers can only be zero if one of them are zero.
		            However, the domain was restricted to $\Z^+$, which means neither $a$ or $b$
		            can be 0.

		            Thus, we get a contradiction
		            \begin{align*}
			            \exists a \exists b ( \sqrt{a + b} & = \sqrt a + \sqrt b)   \\
			                                               & \to (a = 0 \lor b = 0)
			            \land \lnot (a = 0 \lor b = 0)
		            \end{align*}

		            $\qed$

	      \end{enumerate}

	\item \begin{enumerate}
		      \item We can prove the following with a witness. Let $x = 1$.

		            \begin{align*}
			            \mult(1) & = \{1 \cdot n \mid n \in \Z^+\} \\
			                     & = \{n \mid n \in \Z^+\}         \\
			                     & = \Z^+
		            \end{align*}

		            We want to show

		            $$
			            \forall a \{ a \in \mult(y) \to a \in \Z^+ \}
		            $$

		            From the definition of set builder notation, all elements of $\mult(y)$
		            are in $\Z^+$, so the second part of the implication is always true.

		            $\qed$

		      \item Counterexample: Let $x=1$. There is no $y \ne 1$ that satisfies the statement.

		      \item Counterexample: Let $x=2$ and $y=3$.
		      \item Towards a proof of universal generalization, let $z = xy$. We want to show
		            $$
			            \forall b ( b \in \mult(xy) \to b \in \mult(x) \land b \in \mult(y))
		            $$

		            where

		            \begin{align*}
			            \mult(x)  & = \{nx \mid n \in \Z^+ \}     \\
			            \mult(y)  & = \{ny \mid n \in \Z^+ \}     \\
			            \mult(xy) & = \{n_1xy \mid n_1 \in \Z^+\}
		            \end{align*}

		            If we let $n = n_1x$, we see that every element of $\mult(xy)$ is in
		            $\mult(y)$ because the integers are closed under
		            multiplication. Similarly if we let $n = n_1y$, we see
		            that every element of $\mult(xy)$ is in $\mult(x)$.

		            $\qed$

	      \end{enumerate}

	\item \begin{enumerate}[(i)]
		      \item We want to show

		            $$
			            \forall a (a \in T \to a \in \Z^+)
		            $$

		            By the definition of set builder notation, the statement is true.

		      \item We want to show

		            $$
			            \exists x \exists b (2x^2 = b^2)
		            $$

		            to show that the set is nonempty. We assume this to be true.

		      \item We want to show

		            $$
			            \forall b (b^2 \text{ is even} \to b \text{ is even})
		            $$

		            We will prove the contrapositive

		            $$
			            \forall b (b \text{ is odd} \to b^2 \text{ is odd})
		            $$

		            By the definition of odd, $b$ can be written as $b = 2n + 1$, with $n \in \Z$.
		            Squaring this quantity, we get

		            \begin{align*}
			            (2n + 1)^2 & = 4n^2 + 4n + 1    \\
			                       & = 2(2n^2 + 2n) + 1 \\
		            \end{align*}

		            Since integers are closed under multiplication, $2n^2 + 2n \in \Z$,
		            so the original statement is true.

		            $\qed$

		      \item Because multiplication by a positive integer increases a number's
		            value, the equation $2c^2 = s^2$ implies $c^2 < s^2$.
		            Because the function $f(x) = x^2$ strictly increases for
		            $x \in \Z^+$, $c^2 < s^2 \to c < s$.

		      \item Since $2c^2 = s^2$, $c \in T$. Because $s$ is the smallest element of
		            $T$, $s \le c$.

		      \item $$
			            r  \equiv c < s \\
		            $$
	      \end{enumerate}
\end{enumerate}
\end{document}
