\documentclass{article}

\usepackage{amsmath,amsfonts}
\hbadness=100000000
\hfuzz=100pt
\title{Homework 2}
\author{
	Nathaniel Thomas\\
	\texttt{A17069898}
	\and
	Brandon Szeto\\
	\texttt{A17002478}
	\and
	Darren Yu\\
	\texttt{A16760914}
}
\begin{document}

\maketitle

\begin{enumerate}
	\item \begin{enumerate}
		      \item \begin{align*}
			             & \textbf{Basis Step: }
			            0 \in S_4, 1 \in S_4, 2 \in S_4, 3 \in S_4                                      \\
			             & \textbf{Recursive Step: }
			            \text{if } s \in S_4 \text{ and } t \in \{0, 1, 2, 3\} \text{ then } st \in S_4 \\
		            \end{align*}

		      \item \begin{align*}
			             & f_4: S_4 \to \mathbb{Z}^+                                                                                                       \\
			             & \textbf{Basis step: }       \text{if } b \in \{0,1,2,3\} \text{ then } f_4(b) = b                                               \\
			             & \textbf{Recursive step: }   \text{if } b \in \{0,1,2,3\} \text{ and } s \in S_4 \text{ then } f_4(sb) = f_4(b) + 4 \cdot f_4(s) \\
		            \end{align*}

		      \item \begin{align*}
			             & p_4: S_4 \to \{0, 1\}^*                                                                                                   \\
			             & \textbf{Basis step: } \text{if } b \in \{0,1,2,3\}                                                                        \\
			             & \text{ then }
			            p_4(b) =
			            \begin{cases}
				            00 \text{ if } b = 0 \\
				            01 \text{ if } b = 1 \\
				            10 \text{ if } b = 2 \\
				            11 \text{ if } b = 3 \\
			            \end{cases}                                                                                                         \\
			            % &
			             & \textbf{Recursive step: } \text{if } s \in S_4 \text{ and } b \in \{0,1,2,3\} \text{ then } p_4(sb) = p_4(s) \circ p_4(b) \\
		            \end{align*}
	      \end{enumerate}

	\item

	      {\bf Excercise 1:} \\
	      \begin{enumerate}
		      \item \begin{tabular}{ccccc}
			             & 45 & \text{ mod } & 12 & = \underline{9} \\
			             & 45 & -            & 9  & = 36            \\
			             & 36 & \text{ div } & 12 & = \underline{3} \\
			             &    &              &    & \boxed{39}      \\
		            \end{tabular}

		      \item \begin{tabular}{ccccc}
			             & 999 & \text{ mod } & 12 & = \underline{3}  \\
			             & 999 & \text{ div } & 12 & = 83             \\
			             & 83  & \text{ mod } & 12 & = \underline{11} \\
			             & 83  & \text{ div } & 12 & = 6              \\
			             & 6   & \text{ mod } & 12 & = \underline{6}  \\
			             &     &              &    & \boxed{6E3}      \\
		            \end{tabular}

		      \item \begin{tabular}{ccccc}
			             & 322 & \text{ mod } 12 & = \underline{10} \\
			             & 322 & \text{ div } 12 & = 26             \\
			             & 26  & \text{ mod } 12 & = \underline{2}  \\
			             & 26  & \text{ div } 12 & = 2              \\
			             & 2   & \text{ mod } 12 & = \underline{2}  \\
			             &     &                 & \boxed{22X}      \\
		            \end{tabular}

		            \begin{tabular}{cccc}
			             & X $\times$ 12 + X $\times$ 1                                    & = & \boxed{130}  \\
			             & 2 $\times$ 1728 + 0 $\times$ 144 + 2 $\times$ 12 + 2 $\times$ 1 & = & \boxed{3482} \\
			             & E $\times$ 144 + 0 $\times$ 12 + 0 $\times$ 1                   & = & \boxed{1584} \\
		            \end{tabular}
	      \end{enumerate}

	      {\bf Exercise 2: } \\
	      \begin{enumerate}
		      \item True if the last digit $\in \{0, 2,4,6,8,X\}$.
		      \item True if the last digit $\in \{0, 3,6,9\}$.
		      \item True if the last digit $\in \{0, 4,8\}$.
		      \item True if the last digit $\in \{0, 6\}$.
		      \item True if the last two digits $\in \{0\}$.
		      \item True if the second-to-last digit is divisible by 2 and the last digit $\in \{0, 8\}$ or
		            the second-to-last digit is not divisible by 2 and the last digit $ = 4$.
		            % \item True if the last digit $\in \{0,2,4,6,8,X\}$ and the number divided by 2 has the last digit
		            %       $\in \{0, 4,8\}$ and the number $< 8$ * \\

		      \item Recursively sum the digits of the number until the result has 1 dozenal digit. If that
		            digit is E, then the number is divisible by 11.

		      \item True if the number ends with the digit 0.
	      \end{enumerate}

	      {\bf Exercise 3: } \\

	      \begin{tabular}{ccccccc}
		        & 1 &   & 1 & 1 & 1 &   \\
		        &   & 9 & 5 & 8 & E & E \\
		      + &   & E & 5 & X & 7 & X \\
		      \hline
		        & 1 & 8 & E & 7 & 7 & 9 \\
	      \end{tabular} \\
	      \boxed{1  8  E  7  7  9} \\ \\ \\


	      {\bf Exercise 4: } \\

	      \begin{tabular}{ccccc}
		               &   &   & 1 &   \\
		               &   &   & E & 2 \\
		      $\times$ &   &   & 4 & X \\
		      \hline
		               & 1 &   &   &   \\
		               &   & 9 & 3 & 8 \\
		      +        & 3 & 8 & 8 & 0 \\
		      \hline
		               & 4 & 5 & E & 8 \\
	      \end{tabular}

	      \boxed{45E8} \\

	      {\bf Exercise 5: } \\

	      \begin{enumerate}
		      \item 0;8
		      \item 0;2
		      \item 0;26
		      \item 0;368
	      \end{enumerate}

	      {\bf Exercise 6: } \\
	      \begin{enumerate}
		      \item If you are a bakery owner and all of the inventory counts are multiples of 12.
		      \item If your daytime job is counting human toes.
	      \end{enumerate}

	\item
	      % n^b - 1 is the max
	      % 42445 = n^b - 1
	      % b = log_n(42446)
	      {\bf Exercise 1: } \\

	      \begin{tabular}{ccccccc}
		      2: & 2 $\times \log_2(42445 + 1) + 1$  = & 2 $\times$ 16 & = \$ 32 \\
		      3: & 3 $\times \log_3(42445 + 1) + 1$  = & 3 $\times$ 10 & = \$ 30 \\
		      4: & 4 $\times \log_4(42445 + 1) + 1$  = & 4 $\times$ 8  & = \$ 32 \\
		      5: & 5 $\times \log_5(42445 + 1) + 1$  = & 5 $\times$ 7  & = \$ 35 \\
		      6: & 6 $\times \log_6(42445 + 1) + 1$  = & 6 $\times$ 6  & = \$ 36 \\
		      7: & 7 $\times \log_7(42445 + 1) + 1$  = & 7 $\times$ 6  & = \$ 42 \\
		      8: & 8 $\times \log_8(42445 + 1) + 1$  = & 8 $\times$ 6  & = \$ 48 \\
		      9: & 9 $\times \log_9(42445 + 1) + 1$  = & 9 $\times$ 5  & = \$ 45 \\
	      \end{tabular}

	      The cheapest one is the tally counter with wheel size 3. \\

	      {\bf Exercise 2: } \\

	      The (continuous) cost is represented by the function: \\

	      \begin{align*}
		      C(x, n) = \frac{x \log(n + 1)}{\log(x)} \\
	      \end{align*}

	      Where $x$ is the length of the wheel, and $n$ is the maximum number the wheels need to store.
	      To find the base at which the cost is minimized for any given $n$, we take the
	      derivative of $C$ with respect to $x$ and set it to $0$.

	      \begin{align*}
		      \frac{ \partial }{\partial x} \left( \frac{x \ln(n+1)}{\ln(x)} \right)  & = 0                                       \\
		      \ln(n+1)                                                                & = \frac{\ln(n+1)}{\ln(x)}                 \\
		      \ln(x)                                                                  & = 1                                       \\
		      \implies                                                              x & = e \approx 2.718 \to 3 \text{ (rounded)}
	      \end{align*}

	      We see that the cost will be minimized at wheel size $ = 3$,
	      regardless of the maximum number that needs to be represented.

	      Both the size of the base and the amount of space that storing data usees
	      have costs that must be accounted for when building computers. Base 2 is
	      very cost effective. However, if the cost equation for computers is similar to
	      the one above, base 3 may be a better choice. \\



	\item
	      \begin{enumerate}
		      \item (83, 218, 199)
		      \item (96, C8, FA)
		      \item 98967F
		      \item \begin{enumerate}
			            \item A3B \\
			            \item % 502393 $\to$ $(07AA79)_{ 16 }$
			                  \begin{tabular}{cccc}
				                  502393 & \textbf{mod} & 16  = & 9     \\
				                  502393 & \textbf{div} & 16  = & 31399 \\
				                  31399  & \textbf{mod} & 16  = & 7     \\
				                  31399  & \textbf{div} & 16  = & 1962  \\
				                  1962   & \textbf{mod} & 16  = & A     \\
				                  1962   & \textbf{div} & 16  = & 122   \\
				                  122    & \textbf{mod} & 16  = & A     \\
				                  122    & \textbf{div} & 16  = & 7     \\
				                  7      & \textbf{mod} & 16  = & 7     \\
			                  \end{tabular} \\
			                  CC(07AA79) = \boxed{0A7} \\
		            \end{enumerate}
	      \end{enumerate}

	\item \begin{enumerate}
		      \item
		            01010001  \\
		            11110101\\
		            % \item 0110 \\1100
		            10010100\\
		      \item
		            \begin{align*}
			            (10101010)_{2c,8} & = -(256 - 170) = -86  \\
			            (11011011)_{2c,8} & = -(256 - 219) = -37  \\
			            (10001001)_{2c,8} & = -(256 - 137) = -119 \\
		            \end{align*}
		      \item \begin{enumerate}

			            \item \begin{tabular}{cccccccccccc}
				                   & 1  & 1 & 1 & 1 & 1 & 1 & 1 &   &   \\
				                   &    & 0 & 0 & 0 & 1 & 1 & 1 & 1 & 0 \\
				                   & +  & 1 & 1 & 1 & 0 & 0 & 0 & 1 & 0 \\
				                  \hline
				                   & \_ & 0 & 0 & 0 & 0 & 0 & 0 & 0 & 0 \\
			                  \end{tabular}

			                  \boxed{00000000}

			            \item
			                  \begin{align*}
				                  T_8(-19) = ( 11101101 )_{2c,8} \\
				                  T_8(-93) = ( 10100011 )_{2c,8} \\
			                  \end{align*}
			                  \begin{tabular}{cccccccccccc}
				                   & 1  & 1 & 1 &   & 1 & 1 & 1 & 1 &   \\
				                   &    & 1 & 1 & 1 & 0 & 1 & 1 & 0 & 1 \\
				                   & +  & 1 & 0 & 1 & 0 & 0 & 0 & 1 & 1 \\
				                  \hline
				                   & \_ & 1 & 0 & 0 & 1 & 0 & 0 & 0 & 0 \\
			                  \end{tabular}

			                  \boxed{1  0  0  1  0  0  0  0} \\

			            \item
			                  \begin{tabular}{cccccccccccccccccccc}
				                         &          &    &    &    &    &    &    &    & 1 & 1 & 1 & 1 & 0 & 0 & 0 & 1 \\
				                         & $\times$ &    &    &    &    &    &    &    & 0 & 0 & 0 & 0 & 0 & 1 & 0 & 1 \\
				                  \hline &          &    &    &    &    &                                              \\
				                         &          &    &    &    &    &    &    &    & 1 & 1 & 1 & 1 & 0 & 0 & 0 & 1 \\
				                         &          &    &    &    &    &    &    & 0  & 0 & 0 & 0 & 0 & 0 & 0 & 0 & 0 \\
				                         &          &    &    &    &    &    & 1  & 1  & 1 & 1 & 0 & 0 & 0 & 1 & 0 & 0 \\
				                         &          &    &    &    &    & 0  & 0  & 0  & 0 & 0 & 0 & 0 & 0 & 0 & 0 & 0 \\
				                         &          &    &    &    & 0  & 0  & 0  & 0  & 0 & 0 & 0 & 0 & 0 & 0 & 0 & 0 \\
				                         &          &    &    & 0  & 0  & 0  & 0  & 0  & 0 & 0 & 0 & 0 & 0 & 0 & 0 & 0 \\
				                         &          &    & 0  & 0  & 0  & 0  & 0  & 0  & 0 & 0 & 0 & 0 & 0 & 0 & 0 & 0 \\
				                         & +        & 0  & 0  & 0  & 0  & 0  & 0  & 0  & 0 & 0 & 0 & 0 & 0 & 0 & 0 & 0 \\
				                  \hline
				                         &          & \_ & \_ & \_ & \_ & \_ & \_ & \_ & 1 & 0 & 1 & 1 & 0 & 1 & 0 & 1 \\
			                  \end{tabular}
			                  $$
				                  (10110101)_{2c,8} = -(256 - 181) = \boxed{ -75 }
			                  $$
		            \end{enumerate}

	      \end{enumerate}


\end{enumerate}

\end{document}
